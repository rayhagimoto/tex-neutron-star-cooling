%===
% SECTION: Introduction
%===
\section{Introduction}
\label{sec:introduction}
The goal of this project is to evaluate the integral 

%---
% emissivity integral
%---
\begin{bluenv}{Emissivity integral}
    \begin{equation}
    \label{eq:emissivity-integral}
    \begin{aligned}
        \varepsilon_{3'} = \int 
            & f_1 f_2 (1 - f_{1'}) (1 - f_{2'}) (1 + f_{3'}) \\
            &\times E_{3'} \,
            \sum_{\sigma, \sigma'} | \mathcal{M} |^2 \,
            \dd \Phi_3 \big( (p_1 + p_2)^2; 1', 2', 3'\big)
            \,
            \frac{\dd^3 p_1}{(2\pi)^3\,2 E_1}
            \frac{\dd^3 p_2}{(2\pi)^3\,2 E_2} 
            \; 
    \end{aligned}
    \end{equation}
\end{bluenv}
where, 
\begin{bluenv}{Lorentz invariant phase space measure}
    \begin{equation}
        \label{eq:LIPS}
        \dd \Phi_n \big((p_a + p_b)^2; 1, \ldots, n \big) \equiv 
        \prod_{i=1}^{n}
        \bigg[
             \delta (p_i^2 - m_i^2) \, \Theta(p_i^0) \, 
            \frac{\dd^4 p_i}{(2\pi)^3}
        \bigg] \, (2\pi)^4 \delta^4 
        \bigg( 
            p_a + p_b - \sum_{i=1}^{n} p_i
        \bigg) \quad 
    \end{equation}
\end{bluenv}
is the Lorentz-invariant phase space measure, and
\begin{equation*}
    f_i \equiv f_{FD}(E_i) \equiv \frac{1}{e^{(E_i - \mu_i)/T} + 1}
\end{equation*}
is the Fermi-Dirac distribution. 
The spin-summed matrix element squared is given by
\begin{bluenv}{Spin-summed matrix element}
    \begin{equation}
        \label{eq:matrix-element}
        \sum_{\sigma, \sigma'} | \mathcal{M} |^2 
            = 
            \frac{128 \, g^2_{ae\mu} e^4}{(m_1^2 - m_{1'}^2)^2} \,
            \frac{
                 (p_1 \cdot p_{1'} - m_1 m_{1'})
                (p_2 \cdot p_{3'})
                (p_{2'} \cdot p_{3'})
            }
            {
                (p_2 - p_{2'})^4
            } \; \, .
    \end{equation}
\end{bluenv}


%====
% SUBSECTION: Direct integration
%====
\subsec{Integration in terms of momentum variables (didn't work)}
\label{subsec:direct-momentum-integration}
\emph{The first strategy I used to evaluate this integral was to do the integration directly in terms of the momentum variables. Below are the notes I wrote detailing my strategy.}

We can use the momentum conserving Dirac delta to evaluate the $\bm{p}_{2'}$ integral, setting
\begin{align}
    \bm{p}_{2'} = \bm{p}_{1} + \bm{p}_{2} - \bm{p}_{1'} - \bm{p}_{3'}.
\end{align}
We now choose to align the $z$-axis with $\bm{p}_1$ and measure angles with respect to this axis. 
Converting to spherical polar coordinates gives, for example, $\dd^3 p_{3'} = p_{3'}^2 \, \dd p_{3'} \, \dd \cos \theta_{13'} \, \dd \phi_{13'}$
\footnote{The angle $\phi_{13'}$ is not measured with respect to $\bm{p}_1$, it is measured to some axis orthogonal to $\bm{p}_1$. We don't need to define that axis explicitly as long as the other angles $\phi_{1i}$ are measured with respect to the same axis.}
. 
Then the energy conserving Dirac delta can be used to evaluate the $\dd p_{3'}$ integral in the following way. 
First, we use $E = \sqrt{p^2 + m^2}$ to rewrite the masses. 
Then we use momentum conservation to make the replacement $\bm{p}_{2'} \rightarrow \bm{p}_{1} + \bm{p}_{2} - \bm{p}_{1'} - \bm{p}_{3'}$:
\begin{align*}
    &E_1 + E_2 
    = E_{1'} + E_{2'} + E_{3'}\\
    &\Rightarrow \sqrt{p_1^2 + m_1^2} + \sqrt{p_{2}^2 + m_2^2} =
        \sqrt{p_{1'}^2 + m_{1'}^2} + 
        \sqrt{p_{2'}^2 + m_{2'}^2} + 
        \sqrt{p_{3'}^2 + m_{3'}^2}\\
    &\Rightarrow \sqrt{p_1^2 + m_1^2} + \sqrt{p_{2}^2 + m_2^2} =
        \sqrt{p_{1'}^2 + m_{1'}^2} + 
        \sqrt{(\bm{p}_{1} + \bm{p}_{2} - \bm{p}_{1'} - \bm{p}_{3'})^2 + m_{2'}^2} + 
        \sqrt{p_{3'}^2 + m_{3'}^2}. \numberthis{} \label{eq:p3p-equation-1}
\end{align*} 
We would like to solve eqn. (\ref{eq:p3p-equation-1}) for $p_{3'}$, but there is a problem coming from the fact that $p_{3'}$ appears under two square root symbols which makes it impossible to get an expression of the form $p_{3'} = \cdots$. 
Instead, we will make the approximation that because $p_{3'} \ll p_{i}$ for $i \in \{1, 2, 1', 2'\}$ that we can ignore the $p_{3'}$ that shows up in $\sqrt{(\bm{p}_{1} + \bm{p}_{2} - \bm{p}_{1'} - \bm{p}_{3'})^2 + m_{2'}^2}$~. 
The result we obtain when solving for $p_{3'}$ is then
\begin{equation}
\begin{aligned}
    p_{3'} = 
        \bigg[
            &m_1^2 + m_{1'}^2 + m_2^2 + m_{2'}^2 - m_{3'}^2 + 2 p_{1}^2 - 2 p_{1} p_{1'} c_{11'} + 2 p_{1'}^2 - 2 E_1 E_{1'} \\
            &+ 2 p_1 p_2 c_{12} - 2 p_{2} p_{1'} c_{21'} + 2 p_2^2 + 2 E_1 - 2 E_{1'} E_2 \\
            &- 2 E_1 E_{2'}(p_1, p_{2}, p_{1'}, c_{11'}, c_{12}, c_{21'})
            + 2 E_{1'} E_{2'}(p_1, p_{2}, p_{1'}, c_{11'}, c_{12}, c_{21'})\\
            &- 2 E_2 E_{2'}(p_1, p_{2}, p_{1'}, c_{11'}, c_{12}, c_{21'})
        \bigg]^{1/2}
\end{aligned}
\end{equation}
where $E_i \equiv \sqrt{p_i^2 + m_i^2}$, $c_{ij} \equiv \cos \theta_{ij} \equiv \bm{p}_i \cdot \bm{p}_j / (p_i p_j)$ and 
\begin{align}
    E_{2'}(p_1, p_{2}, p_{1'}, c_{11'}, c_{12}, c_{21'}) = \sqrt{p_1^2  - 2 p_1 p_{1'} c_{11'} + p_{1'}^2  + 2 p_{1} p_{2} c_{12} - 2 p_2 p_{1'} c_{21'} + p_2^2 + m_{2'}^2}.    
\end{align}
Now that we have an expression for $p_{3'}$ the requirement that $p_{3'}$ must be real and non-negative restricts the domain of integration of the other variables. 
\hl{I am not sure how to derive the new limits so I got stuck here.}


% ====
% SUBSECTION: Wisdom from Weinberg
% ==== 
\subsec{Weinberg's wisdom}
\label{subsec:Weinberg}

\hl{Update: 11 Aug 2023:} I talked to Hong-Yi today and he pointed out page 141 of Weinberg QFT vol. I where the following treatment is done for the outgoing particle phase space measure.

\begin{enumerate}
    \item Use momentum conservation to eliminate one of the momenta, e.g. $\pvec[1]$.
    \item Rewrite integrals over other two momenta in spherical polar coords so that $\dd^3 p_2 \, \dd^3 p_3 = \pmag[2]^2 \dd \pmag[2] \, \pmag[3]^2 \dd \pmag[3] \dd \Omega_3 \, \dd \phi_{23} \, \dd \cos \theta_{23}$ .
    \item $\cos \theta_{23}$ is then fixed by energy conservation:
        \begin{align}
            \sqrt{
                \pmag[2]^2 + 2\pmag[2]\pmag[3] \cos \theta_{23} + \pmag[3]^2 + m_1^2
            }
            + \sqrt{\pmag[2]^2 + m_2^2} + \sqrt{\pmag[3]^2 + m_3^2} = E .
        \end{align}
    \item This yields $\delta^4 ( p_{\mathrm{in}} - p_{\mathrm{out}}) \dd \beta \rightarrow \pmag[2] \dd \pmag[2] \pmag[3] \dd \pmag[3] E_1 \dd \Omega_3 \dd \phi_{23}$. Rewriting in terms of energies gives $E_1 E_2 E_3 \dd E_2 \dd E_3 \dd \Omega_{3} \dd \phi_{23}$ .
\end{enumerate}