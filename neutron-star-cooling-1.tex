%===
% SECTION: Introduction
%===
\section{Introduction}
\label{sec:introduction}
The goal of this project is to evaluate the integral 

%---
% emissivity integral
%---
\begin{bluenv}{Emissivity integral}
    \begin{equation}
    \label{eq:emissivity-integral}
    \begin{aligned}
        \varepsilon_3 = \int 
            & f_a f_b (1 - f_1) (1 - f_2) (1 + f_3) \\
            &\times E_3 \,
            \overline{| \mathcal{M} |^2} \,
            \dd \Phi_3 \big( (p_a + p_b)^2; m_1, m_2, m_3 \big)
            \,
            \frac{\dd^3 p_a}{(2\pi)^3\,2 E_a}
            \frac{\dd^3 p_b}{(2\pi)^3\,2 E_b} 
            \; 
    \end{aligned}
    \end{equation}
\end{bluenv}
where, 
\begin{bluenv}{Lorentz invariant phase space measure}
    \begin{equation}
        \label{eq:LIPS}
        \dd \Phi_n \big((p_a + p_b)^2; 1, \ldots, n \big) \equiv 
        \prod_{i=1}^{n}
        \bigg[
             \delta (p_i^2 - m_i^2) \, \Theta(p_i^0) \, 
            \frac{\dd^4 p_i}{(2\pi)^3}
        \bigg] \, (2\pi)^4 \delta^4 
        \bigg( 
            p_a + p_b - \sum_{i=1}^{n} p_i
        \bigg) \quad 
    \end{equation}
\end{bluenv}
is the Lorentz-invariant phase space measure, and
\begin{align}
    f_i &\equiv f_{FD}(E_i) \equiv \frac{1}{e^{(E_i - \mu_i)/T} + 1} \text{ for } i = a, b, 1, 2 \label{eq:fermi-dirac}\\
    f_3 &\equiv f_{BE}(E_i) \equiv \frac{1}{e^{(E_3 + \mu_3)/T} + 1} \label{eq:bose-einstein}
\end{align}
are thermal distributions which show up due to Pauli blocking and Bose enhancement. 
$\overline{|\mathcal{M}|^2}$ depends on the axion emission channel: (i) $l + p \rightarrow l' + p + a$, (ii) $l + l \rightarrow l' + l + a$, (iii) $l + l' \rightarrow l + l + a$, where $l, l'$ denote either $e$ or $\mu$, and $l' \neq l$.
The matrix elements are
\begin{bluenv}{Spin-summed matrix element}
\vspace{-7px}
\begin{align}
    \label{eq:lp-matrix-element}
    \overline{| \mathcal{M}^{(lp)} |^2} 
    &= 
    - \frac{128 \, g^2_{ae\mu} e^4}{(m_a^2 - m_1^2)^2} \,
        \frac{
            (p_a \cdot p_1 + m_a m_1)
            (p_b \cdot p_3)
            (p_2 \cdot p_3)
        }
        {
            (p_b - p_2)^4
        } \\
    \label{eq:ll-matrix-element}
    \overline{| \mathcal{M}^{(ll)} |^2} 
        &= 
        \overline{| \mathcal{M}^{(lp)} |^2} 
        + (a \leftrightarrow b) 
        + \mathcal{T}^{(ll)} \\
    \label{eq:llp-matrix-element}
    \overline{| \mathcal{M}^{(ll')} |^2} 
        &= 
        \overline{| \mathcal{M}^{(lp)} |^2} 
        + (1 \leftrightarrow 2) 
        + \mathcal{T}^{(ll')} 
\end{align}
\begin{align*}
    \mathcal{T}^{(ll)} 
    = 
    &\frac{64 g_{ae\mu}^2 e^4}{(m_a^2 - m_1^2)^2} \, (p_2 \cdot p_3) \, \times\\
    &\frac{(p_b \cdot p_1 + m_b m_1)(p_a \cdot p_3) + (p_a \cdot p_1 + m_a m_1)(p_b \cdot p_3) - (p_a \cdot p_b + m_a m_b) (p_1 \cdot p_3) }{(p_a  - p_2)^2 \, (p_b - p_2)^2}
\end{align*}
$\mathcal{T}^{(ll')}$ is given by $\mathcal{T}^{(ll)}$ with the replacements
$a \leftrightarrow 1$ and $b \leftrightarrow 2$.

We use a mostly plus metric signature.
\end{bluenv}

Some comments before we move on:
\begin{enumerate}
    \item The axions (denoted in this introduction by $3$) are effectively free-streaming so the Bose-enhancement factor $(1 + f_3)$ will be ignored.
    \item The thermal factors $f_i$ are not Lorentz invariant (unlike the spin-summed matrix element and the integration measure) so we emphasize that \hl{the energies appearing in eqs. {(\ref{eq:fermi-dirac})} and  {(\ref{eq:bose-einstein})} are measured in the rest frame of the neutron star}. This fact will be very important to remember later.
\end{enumerate}

%====
% SUBSECTION: Direct integration
%====
\subsec{Integration in terms of momentum variables (didn't work)}
% \label{subsec:direct-momentum-integration}
\emph{The first strategy I used to evaluate this integral was to do the integration directly in terms of the momentum variables. Below are the notes I wrote detailing my strategy.}

We can use the momentum conserving Dirac delta to evaluate the $\bm{p}_2$ integral, setting
\begin{align}
    \bm{p}_2 = \bm{p}_a + \bm{p}_b - \bm{p}_1 - \bm{p}_3.
\end{align}
We now choose to align the $z$-axis with $\bm{p}_a$ and measure angles with respect to this axis. 
Converting to spherical polar coordinates gives, for example, $\dd^3 p_3 = p_3^2 \, \dd p_3 \, \dd \cos \theta_{13'} \, \dd \phi_{13'}$
. 
Then the energy conserving Dirac delta can be used to evaluate the $\dd p_3$ integral in the following way. 
First, we use $E = \sqrt{p^2 + m^2}$ to rewrite the masses. 
Then we use momentum conservation to make the replacement $\bm{p}_2 \rightarrow \bm{p}_a + \bm{p}_b - \bm{p}_1 - \bm{p}_3$:
\begin{align*}
    &E_a + E_b 
    = E_1 + E_2 + E_3\\
    &\Rightarrow \sqrt{p_a^2 + m_a^2} + \sqrt{p_b^2 + m_b^2} =
        \sqrt{p_1^2 + m_1^2} + 
        \sqrt{p_2^2 + m_2^2} + 
        \sqrt{p_3^2 + m_3^2}\\
    &\Rightarrow \sqrt{p_a^2 + m_a^2} + \sqrt{p_b^2 + m_b^2} =
        \sqrt{p_1^2 + m_1^2} + 
        \sqrt{(\bm{p}_a + \bm{p}_b - \bm{p}_1 - \bm{p}_3)^2 + m_2^2} + 
        \sqrt{p_3^2 + m_3^2}. \numberthis{} \label{eq:p3p-equation-1}
\end{align*} 
We would like to solve eqn. (\ref{eq:p3p-equation-1}) for $p_3$, but there is a problem coming from the fact that $p_3$ appears under two square root symbols which makes it impossible to get an expression of the form $p_3 = \cdots$. 
Instead, we will make the approximation that because $p_3 \ll p_{i}$ for $i \in \{1, 2, 1', 2'\}$ that we can ignore the $p_3$ that shows up in $\sqrt{(\bm{p}_a + \bm{p}_b - \bm{p}_1 - \bm{p}_3)^2 + m_2^2}$~. 
The result we obtain when solving for $p_3$ is then
\begin{equation}
\begin{aligned}
    p_3 = 
        \bigg[
            &m_a^2 + m_1^2 + m_b^2 + m_2^2 - m_3^2 + 2 p_a^2 - 2 p_a p_1 c_{11'} + 2 p_1^2 - 2 E_a E_1 \\
            &+ 2 p_a p_b c_{12} - 2 p_b p_1 c_{21'} + 2 p_b^2 + 2 E_a - 2 E_1 E_b \\
            &- 2 E_a E_2(p_a, p_b, p_1, c_{11'}, c_{12}, c_{21'})
            + 2 E_1 E_2(p_a, p_b, p_1, c_{11'}, c_{12}, c_{21'})\\
            &- 2 E_b E_2(p_a, p_b, p_1, c_{11'}, c_{12}, c_{21'})
        \bigg]^{1/2}
\end{aligned}
\end{equation}
where $E_i \equiv \sqrt{p_i^2 + m_i^2}$, $c_{ij} \equiv \cos \theta_{ij} \equiv \bm{p}_i \cdot \bm{p}_j / (p_i p_j)$ and 
\begin{align}
    E_2(p_a, p_b, p_1, c_{11'}, c_{12}, c_{21'}) = \sqrt{p_a^2  - 2 p_a p_1 c_{11'} + p_1^2  + 2 p_a p_b c_{12} - 2 p_b p_1 c_{21'} + p_b^2 + m_2^2}.    
\end{align}
Now that we have an expression for $p_3$ the requirement that $p_3$ must be real and non-negative restricts the domain of integration of the other variables. 
\hl{I am not sure how to derive the new limits so I got stuck here.}


% % ====
% % SUBSECTION: Wisdom from Weinberg
% % ==== 
% \subsec{Weinberg's wisdom}
% \label{subsec:Weinberg}

% \hl{Update: 11 Aug 2023:} I talked to Hong-Yi today and he pointed out page 141 of Weinberg QFT vol. I where the following treatment is done for the outgoing particle phase space measure.

% \begin{enumerate}
%     \item Use momentum conservation to eliminate one of the momenta, e.g. $\pvec[1]$.
%     \item Rewrite integrals over other two momenta in spherical polar coords so that $\dd^3 p_b \, \dd^3 p_3 = \pmag[2]^2 \dd \pmag[2] \, \pmag[3]^2 \dd \pmag[3] \dd \Omega_3 \, \dd \phi_{23} \, \dd \cos \theta_{23}$ .
%     \item $\cos \theta_{23}$ is then fixed by energy conservation:
%         \begin{align}
%             \sqrt{
%                 \pmag[2]^2 + 2\pmag[2]\pmag[3] \cos \theta_{23} + \pmag[3]^2 + m_a^2
%             }
%             + \sqrt{\pmag[2]^2 + m_b^2} + \sqrt{\pmag[3]^2 + m_3^2} = E .
%         \end{align}
%     \item This yields $\delta^4 ( p_{\mathrm{in}} - p_{\mathrm{out}}) \dd \beta \rightarrow \pmag[2] \dd \pmag[2] \pmag[3] \dd \pmag[3] E_a \dd \Omega_3 \dd \phi_{23}$. Rewriting in terms of energies gives $E_a E_b E_3 \dd E_b \dd E_3 \dd \Omega_{3} \dd \phi_{23}$ .
% \end{enumerate}