%====
% SECTION: SOFTWARE PACKAGES 
%====
\section{CompHEP and MADGRAPH5}
\label{sec:software}

%----
% SUBSECTION: CompHEP
%----
\subsec{\code{CompHEP}}
\label{subsec:comphep}
\newcommand{\comphep}[1][]{\code{CompHEP}}
While I was looking for ways to do the integral I found a software called \comphep~\cite{Boos:1994xb} which does a lot of the things I am interested in doing. 
\comphep is a ``a package for evaluation of Feynman diagrams, integration over multi-particle phase space and event generation''. 
It allows the user to specify a Lagrangian and then it can generate tree level Feynman diagrams for a user-specified process, along with C and Fortran code to evaluate matrix elements. 
It can even compute cross-sections/other observables.
\comphep has two main backbones: (i) symbolic calculations (ii) numerical calculations.

The main tasks performed by the symbolic part are
\begin{enumerate}[itemsep=-2ex]
    \item Select a process by specifying in and out states for decays into up to $5$ outgoing particles or collisions of $2 \rightarrow 2, \  2 \rightarrow 3, \  2 \rightarrow 4$.
    \item Generate Feynman diagrams.
    \item Delete unwanted diagrams.
    \item Compute S-matrix elements.
    \item Save symbolic results so they can be further analysed in \code{Mathematica}.
    \item Generate C and Fortran code for the matrix elements to be used in numerical studies.
\end{enumerate}

The numerical part, on the other hand, performs Monte Carlo integration and event generation. 
The main tasks performed are:
\begin{enumerate}[itemsep=-2ex]
    \item Choose phase space kinematic variables (\hl{idk what this means yet})
    \item Introduce kinematic cuts over squared momentum transfers (\hl{not sure what the significance of this is})
    \item Perform regularization to remove sharp peaks in matrix elements.
    \item Calculate distributions, cross-sections, or particle width using Monte Carlo methods.
    \item Perform integration, taking into account structure functions for ingoing particles.
    \item Event generation.
\end{enumerate}

