\section{Lorentz Transformations}
I implement Lorentz boosts in arbitrary directions in the following way. A boost in the $+z$-direction that takes the vector $(m, 0, 0, 0)$ into $(E, 0, 0, |\pvec|)$ is given by 
\begin{equation}
    \Lambda \big(|\pvec|, m \big) =
    \begin{pmatrix}
        \gamma & 0 & 0 & \sqrt{\gamma^2 - 1} \\
        0 & 1 & 0 & 0 \\
        0 & 0 & 1 & 0 \\
        \sqrt{\gamma^2 - 1} & 0 & 0 & \gamma
    \end{pmatrix}  \ ,
\end{equation}
where $\gamma = \sqrt{|\pvec|^2 + m^2} / m$.
Meanwhile, a rotation that takes the $z$-axis $(\,\cdot\,,\, 0,\, 0,\, 1)$ into the $(\theta, \phi)$ direction is given by 
\begin{equation}
    R(\theta, \phi) = 
    \begin{pmatrix}
        1 & 0 & 0 & 0 \\
        0 & \cos \theta \cos \phi & - \sin \phi & \cos \phi \sin \theta \\
        0 & \cos \theta \sin \phi & \cos \phi & \sin \theta \sin \phi \\
        0 & - \sin \theta & 0 & \cos \theta
    \end{pmatrix} \ .
\end{equation}
Thus, a boost by momentum $|\pvec|$ in the direction $\hat{\bm{n}} \equiv (\theta, \phi)$ is given by $\Lambda\big(|\pvec|, m, \hat{\bm{n}}\big) \equiv R(\theta, \phi) \, \Lambda \big(|\pvec|, m \big) \, R^{-1}(\theta, \phi)$. 
I had Mathematically symbolically evaluate this matrix and use that to evaluate boosts on four-vectors. 

\subsec{Boosting in-state momenta}
\label{subsec:boosting-in-state-momenta}
Suppose the components of $p_a$ and $p_b$ as seen in the rest-frame of the neutron star are known.
We will simply call this frame the NS frame and denote quantities as seen in this frame with a superscript.
In order to make use of the momentum transfer method I need to perform a boost and rotation to obtain the components in a frame where $p_a = (m_a, \bm{0})$ and $\pvec[b] \propto - \hat{\mathbf{z}}$.
We will refer to this frame as the rest frame of $a$, or simply just the `$a$ frame', denoting quantities as seen in this frame by a superscript $\mathrm{a}$.
This is trivial for $p_a$: simply set $p_a = (m_a, \bm{0})$. 
For $p_b$ we need only boost it by applying $\Lambda^{-1}\big(|\pvec[a]|, m_b, \hat{\bm{n}}_a \big)$ with $\bm{n}_a = (\theta_a, \phi_a)$ being the spherical coordinates of the direction of $\pvec[a]$. 
Then, we can manually set $p_b = (E_b, 0, 0, -|\pvec[b]|)$. 
Doing things this way helps reduce the room for matrix multiplication to introduce numerical errors in our results.

In addition to this, recall that the out-state momenta $p_i$ will be generated from the integration variables $s_i, t_i, \varphi_i$ in the rest frame so in order to calculate quantities like $E_3^{\mathrm{NS}}$ we need to boost the $p_i$ vectors back into the neutron star rest frame.
This can be done by calculating $\Lambda \big ( |\pvec[a], m_i, \hat{\bm{n}}_b^{\mathrm{NS}} \big)$ where $\hat{\bm{n}}_b^{\mathrm{NS}}$ gives the direction of $\pvec[b]^{\mathrm{NS}}$. 
In the $a$ frame,
\begin{align}
    p_1^{\mathrm{a}} &= (p_a^\mathrm{a} - Q_1) \\
    p_2^{\mathrm{a}} &= (Q_1 - Q_2) \\
    p_3^{\mathrm{a}} &= (Q_2 + p_b^\mathrm{a}) \ .
\end{align}
These are converted to the NS frame by calculating $\Lambda \big (|\pvec[a]|, m_i, \hat{\bm{n}}\big) \, R(\theta_b^{\mathrm{NS}}, \phi_b^{\mathrm{NS}})$.