%====
% Appendix B: 2 p2.p3
%====
% Macros for App. B:
\newcommand*{\sintheta}[2]{
    \sqrt{\frac{4 m_a^2 (-G_{#1})}{\lambda_{#1} \lambda_{#2}}}
}
\newcommand*{\costheta}[2]{
    \frac{\sqrt{\Lambda_{#1} \Lambda_{#2}} - 2 m_a^2 (t_{#1} + t_{#2} - m_{#2}^2)}{\sqrt{\lambda_{#1}\lambda_{#2}}}
}

\section{Evaluating $2 p_2 \cdot p_3$}
\label{app:2-dot-3}

Let $\varepsilon_i = Q_i^0 = \sqrt{\Qmag{i}^2 + t_i}$. Then,
\begin{equation}
    \label{eq:p2_dot_p3-Minkowski-dot-product}
    \begin{aligned}
        p_2 \cdot p_3 
        &=
            (Q_1 - Q_2)\cdot(Q_2 - Q_3)     \\
        &=
            Q_1 \cdot Q_2 - Q_1 \cdot Q_3 - t_2 + Q_2 \cdot Q_3 \\
        &= 
            \varepsilon_1 \varepsilon_2 - \Qvec{1}\cdot\Qvec{2}
        -
            \varepsilon_1 \varepsilon_3 - \Qvec{1}\cdot\Qvec{3}
        -
            t_2
        +
        \varepsilon_2 \varepsilon_3 - \Qvec{2}\cdot\Qvec{3} \, . 
    \end{aligned}
\end{equation}
From \eref{eq:p2_dot_p3-Minkowski-dot-product} it is clear that we need to know the angles between vectors $\Qvec{1}$ and $\Qvec{2}$, $\Qvec{1}$ and $\Qvec{3}$, and $\Qvec{2}$ and $\Qvec{3}$ in terms of the integration variables $t_1, t_2, s_2, \varphi_1, \varphi_2$. 
We already have formulas for all of these (cf. eqs.~(\ref{eq:B-Qmag-formula}-\ref{eq:B-sintheta-formula})) except for the angle between $\Qvec{1}$ and $\Qvec{3}$ defined by $\cos\theta_{13}$. 
To obtain this formula we first pick to align the $z$-axis with $\Qvec{3}$ so that
\begin{align}
    \Qvec{1} &= \Qmag{1} (\cos \varphi_{13} \sin \theta_{13} \hat{\bm{i}} + \sin \varphi_{13} \sin \theta_{13} \hat{\bm{j}} + \cos \theta_{13} \hat{\bm{k}}) \\
    \Qvec{2} &= \Qmag{2} (\cos \varphi_2 \sin \theta_2 \hat{\bm{i}} + \sin \varphi_2 \sin \theta_2 \hat{\bm{j}} + \cos \theta_2 \hat{\bm{k}}) \\
    \Qvec{3} &= \Qmag{3} \hat{\bm{k}} \; .
\end{align}

\reversemarginpar\marginnote{Note that in~\cite{Byckling:1969sx} there is a typo in their eq. (12b): they are missing a factor of $4$ which is present in my \eref{eq:B-sintheta-formula}.}
\begin{bluenv}{Formulas for angles and magnitudes in terms of integration variables}
    \vspace{-2ex}
    \begin{align}
        \Qmag{i} &= \frac{\sqrt{\lambda_i}}{2 m_a} \label{eq:B-Qmag-formula}\\
        \varepsilon_i &= \frac{\sqrt{\Lambda_i}}{2 m_a} \label{eq:B-epsilon-formula}\\
        \cos \theta_i &\equiv \frac{\Qvec{i}\cdot\Qvec{i+1}}{\Qmag{i}\Qmag{i+1}} 
            =  \frac{\xi_i}{\sqrt{\lambda_i \lambda_{i+1}}} \label{eq:B-costheta-formula}\\
        \sin\theta_i &= \sintheta{i}{i+1} \label{eq:B-sintheta-formula}
    \end{align}
    where
    \begin{align}
        \label{eq:app-lambda-definitions}
        \lambda_i &\equiv \lambda(s_i, t_i, m_a^2) \\
        \Lambda_i &\equiv \lambda_i + 4 m_a^2 t_i = (s_i - t_i - m_a^2)^2 \\
        G_i &\equiv G(t_i, s_{i+1}, s_i, t_{i+1}, m_{i+1}^2, m_a^2) < 0 \\
        \xi_i &\equiv \sqrt{\Lambda_i \Lambda_{i+1}} - 2 m_a^2 (t_i + t_{i+1} - m_{i+1}^2) \; .
    \end{align}
\end{bluenv}

Now we can use these identities to evaluate the dot products $Q_2 \cdot Q_3$, $Q_1 \cdot Q_3$, and $Q_1 \cdot Q_2$:
% Q2 dot Q3
\begin{gather}
    \begin{aligned}
        Q_2 \cdot Q_3 
        &= 
        \frac{\sqrt{\Lambda_2 \Lambda_3}}{4 m_a^2} 
        - \frac{\sqrt{\lambda_2 \lambda_3}}{4 m_a^2} 
        \frac{\sqrt{\Lambda_2 \Lambda_3} - 2m_a^2(t_2 + t_3 - m_3^2)}{\sqrt{\lambda_2 \lambda_3}} \\
        &= \frac{1}{2}(t_2 + t_3 - m_3^2) \ . 
    \end{aligned}
\end{gather}
% Q1 dot Q3
\begin{gather}
    \begin{aligned}
        Q_1 \cdot Q_3 = \varepsilon_1 \varepsilon_3 - \Qmag{1}\Qmag{3} \cos \theta_{13}
    \end{aligned}
\end{gather}
% Q1 dot Q2
\begin{gather}
    \begin{aligned}
        Q_1 \cdot Q_2 
            &= 
            \varepsilon_1 \varepsilon_2
                - \Qmag{1} \Qmag{2} 
                \bigl(
                    \sin \theta_1 \sin \theta_2 \cos(\varphi_1 - \varphi_2) + \cos \theta_1 \cos \theta_2
                \bigr)
    \end{aligned}
\end{gather}
\begin{bluenv}{Main result of this appendix}
\vspace{-3ex}
    \begin{gather}
        \begin{aligned}
            p_2 \cdot p_3 &= 
            \text{something}
        \end{aligned}
    \end{gather}
\end{bluenv}